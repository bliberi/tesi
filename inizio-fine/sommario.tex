% !TEX encoding = UTF-8
% !TEX TS-program = pdflatex
% !TEX root = ../tesi.tex

%**************************************************************
% Sommario
%**************************************************************
\cleardoublepage
\phantomsection
\pdfbookmark{Sommario}{Sommario}
\begingroup
\let\clearpage\relax
\let\cleardoublepage\relax
\let\cleardoublepage\relax

\chapter*{Sommario}

Il presente documento descrive il lavoro svolto durante il periodo di stage, della durata di trecento ore, presso l'azienda \TS\ dalla laureanda Beatrice Liberi. \\ \\
Lo scopo dello stage è stato lo studio di modelli per problemi di scheduling e implementazione di algoritmi euristici che restituiscano una soluzione ammissibile in tempi sufficientemente brevi per una pianificazione in tempo reale. Il progetto ha fornito a \TS\ uno strumento che potrà essere integrato in software di pianificazione da essa sviluppati da applicare in diversi contesti per i quali sia necessaria la risoluzione di un problema di scheduling. \\ \\
Gli obiettivi da raggiungere nel corso dello stage erano molteplici: 
in primo luogo era richiesto di assimilare i concetti di base dei problemi di scheduling, studiare la letteratura scientifica sui modelli e sugli algoritmi euristici più adatti per risolvere gli stessi; in secondo luogo era richiesto di definire formalmente con un modello matematico un problema di scheduling da risolvere, e progettare un algoritmo euristico per trovare una soluzione al problema; terzo ed ultimo obiettivo era l'implementazione dell'algoritmo euristico, integrandolo nelle librerie dell'azienda.

\endgroup			

\vfill

