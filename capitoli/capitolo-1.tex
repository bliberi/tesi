% !TEX encoding = UTF-8
% !TEX TS-program = pdflatex
% !TEX root = ../tesi.tex

%**************************************************************
\chapter{Introduzione}
\label{cap:introduzione}
%**************************************************************

\intro{In questo capitolo vengono brevemente descritte l'azienda \TS\ presso la quale ho svolto lo stage e l'idea dalla quale è nata la necessità del progetto portato a termine durante lo stesso. \\
Viene inoltre presentata la suddivisione della tesi per capitoli e vengono introdotte alcune norme tipografiche che verranno utilizzate di seguito.}

%**************************************************************
\section{L'azienda}

\TS\ è un'azienda che si occupa di trasporti su gomma di merci per conto terzi su mezzi pesanti. Con una flotta di venticinque camion che deve compiere carichi e scarichi in tutta l'Italia centro-settentrionale da coordinare in tempo reale, da qualche anno l'azienda ha cominciato a sviluppare, grazie ad un team di informatici e matematici, un sistema per il controllo della flotta stessa e, soprattutto, per la pianificazione di viaggi, carichi e scarichi in modo da ottimizzare sia l'utilizzo dello spazio disponibile sui camion, sia i chilometri percorsi; il tutto vincolato ai tempi concordati di ritiro e consegna. \\
\TS\ vede questo planning come il primo mattone di un sistema molto più articolato che unirà diversi tipi di servizi per il supporto di tutta la \emph{\gls{supply-chain}}\glsfirstoccur, non solo nel campo dei trasporti.

%**************************************************************
\section{L'idea}

Uno degli ulteriori servizi che \TS\ vuole sviluppare consiste in un software per la gestione della \emph{\gls{sched}}\glsfirstoccur (o \textit{scheduling}) dei turni di lavoro. \\ I \emph{\gls{probsched}}\glsfirstoccur\ ricadono generalmente nella classe \emph{\gls{nphard}}\glsfirstoccur\ e risulta quindi particolarmente difficile trovare delle soluzioni ottime, e spesso anche solo soluzioni ammissibili. Inoltre, esistono diversi contesti in cui i turni devono poter essere proposti in tempi molto rapidi, ad esempio per adeguarsi a cambiamenti durante l’orizzonte di pianificazione. Uno degli utilizzi che \TS\ potrebbe fare di questo software, nel particolare, può essere l'organizzazione delle squadre di meccanici. \\
Lo stage si pone dunque in questo contesto di progettazione di un framework per la risoluzione dei problemi di scheduling.

%**************************************************************
\section{Organizzazione del testo}

\begin{description}
    \item[{\hyperref[cap:descrizione-stage]{Il Capitolo 2}}] descrive in dettaglio lo stage. Ne specifica il progetto da svolgere contestualizzandolo nella realtà aziendale e, definendone i requisiti, gli obiettivi da raggiungere e la pianificazione iniziale.
    
    \item[{\hyperref[cap:definizione-problema]{Il Capitolo 3}}] si approfondisce l'argomento dei problemi di scheduling e si definisce nei dettagli il problema da risolvere durante lo stage.
    
    \item[{\hyperref[cap:analisi-requisiti]{Il Capitolo 4}}] consiste nell'analisi dei requisiti svolta per il progetto, approfondita con diagrammi dei casi d'uso.
    
    \item[{\hyperref[cap:progettazione-codifica]{Il Capitolo 5}}] presenta la progettazione svolta per il progetto, approfondita con diagrammi \emph{\gls{umlg}}\glsfirstoccur, e ne descrive la fase di codifica.
    
    \item[{\hyperref[cap:verifica-validazione]{Il Capitolo 6}}] approfondisce la fase di verifica e validazione del progetto, specificando le modalità ed i risultati ottenuti.
    
    \item[{\hyperref[cap:conclusioni]{Il Capitolo 7}}]
    riporta le conclusioni oggettive e soggettive a cui si è giunti per il progetto.
\end{description}

\noindent
Riguardo la stesura del testo, relativamente al documento sono state adottate le seguenti convenzioni tipografiche:
\begin{itemize}
	\item gli acronimi, le abbreviazioni e i termini ambigui o di uso non comune menzionati vengono definiti nel glossario, situato alla fine del presente documento;
	\item per la prima occorrenza dei termini riportati nel glossario viene utilizzata la seguente nomenclatura: \emph{parola}\glsfirstoccur;
	\item i termini in lingua straniera o facenti parti del gergo tecnico sono evidenziati in \emph{corsivo}.
\end{itemize}